\documentclass[12pt]{article}
\usepackage{amsmath, amssymb}
\usepackage{geometry}
\usepackage{graphicx}
\geometry{margin=1in}

\begin{document}

\begin{minipage}{0.2\columnwidth}
\includegraphics[width=\linewidth]{iiitb_comet_logo.jpg}
\end{minipage}
\hfill
\begin{minipage}{0.8\columnwidth}
\raggedleft
{Name:} PRANJAL ROY\\
{COMET ID:} COMETFWC062\\
{Date:} \today
\end{minipage}

\begin{center}
\large \textbf{Tenth International Olympiad, 1968}
\end{center}

\section* {1968/1.}

Prove that there is one and only one triangle whose side lengths are consecutive integers, and one of whose angles is twice as large as another.

\section* {1968/2.}
Find all natural numbers x such that the product of their digits (in decimal
notation) is equal to $x^2 - 10x - 22$

\section* {1968/3.}

Consider the system of equations

$
\begin{aligned}
ax_1^2 + bx_1 + c = x_2 
\end{aligned}
$

$
\begin{aligned}
    ax_2^2 + bx_2 + c = x_3
\end{aligned} 
$

   ...
   
$
\begin{aligned}
ax_{n-1}^2 + bx_{n-1} + c = x_n
\end{aligned}
$

$
\begin{aligned}
ax_n^2 + bx_n + c = x_1
\end{aligned}
$

with unknowns $x_1, x_2, ..... , x_n$, where $a,b,c$ are real and $a \ne 0$.
Let $\Delta = (b-1)^2 - 4ac$. Prove that for this system

\begin{enumerate}
\item if $\Delta < 0$, there is no solution,
\item if $\Delta = 0$, there is exactly one solution,
\item if $\Delta > 0$, there is more than one solution.
\end{enumerate}


\section* {1968/4.}
Prove that in every tetrahedron there is a vertex such that the three edges
meeting there have lengths which are the sides of a triangle.

\section* {1968/5.}
Let $f$ be a real-valued function defined for all real numbers x such that, for some positive constant a, the equation

$
\begin{aligned}
    f(x+a) = \frac{1}{2} + \sqrt{f(x) - [f(x)]^2}
\end{aligned}
$

\begin{enumerate}
\item[(a)] Prove that the function $f$ is periodic (i.e., there exists a positive number 
$b$ such that $f(x+b)=f(x)$ for all $x$).

\item[(b)] For $a=1$, give an example of a non-constant function with the required
properties.
\end{enumerate}


\section* {1968/6.}
For every natural number n, evaluate the sum

$
\begin{aligned}
\sum_{k=0}^{\infty} \left\lfloor \frac{n + 2^k} {2^{k+1}} \right\rfloor = \left\lfloor \frac{n+1}{2} \right\rfloor + \left\lfloor \frac{n+2}{4} \right\rfloor + ... + \left\lfloor \frac{n+2^k}{2^{k+1}} \right\rfloor
+ ...
\end{aligned}
$
(The symbol [x] denotes the greatest integer not exceeding x.)
\end{document}
