\documentclass[12pt]{article}
\usepackage{amsmath, amssymb}
\usepackage{geometry}
\usepackage{graphicx}
\geometry{margin=1in}

\begin{document}

\begin{minipage}{0.2\columnwidth}
\includegraphics[width=\linewidth]{iiitb_comet_logo.jpg}
\end{minipage}
\hfill
\begin{minipage}{0.8\columnwidth}
\raggedleft
{Name:} PRANJAL ROY\\
{COMET ID:} COMETFWC062\\
{Date:} \today
\end{minipage}


\begin{center}
\large \textbf{Twelfth International Olympiad, 1970}
\end{center}

\section* {1970/1.}

Let $M$ be a point on the side $AB$ of $\triangle ABC$. 
Let $r_1, r_2$ and $r$ be the radii of the inscribed circles of triangles 
$AMC$, $BMC$ and $ABC$. Let $q_1, q_2$ and $q$ be the radii of the 
escribed circles of the same triangles that lie in the angle $ACB$. 
Prove that

$
\begin{aligned}
\frac{r_1}{q_1}\cdot \frac{r_2}{q_2} = \frac{r}{q}.
\end{aligned}
$

\section* {1970/2.}

Let $a, b, n$ be integers greater than $1$, and let $a$ and $b$ be the bases of two number systems. $A_{n-1}$ and $A_n$ are numbers in the system with base $a$, and $B_{n-1}$ and $B_n$ are numbers in the system with base $b$; these are related as follows:

$
\begin{aligned}
A_n = x_n x_{n-1} ... x_0, A_{n-1} = x_{n-1} x_{n-2} ... x_0,
\end{aligned}
$

$
\begin{aligned}
B_n = x_n x_{n-1} ... x_0, B_{n-1} = x_{n-1} x_{n-2} ... x_0,
\end{aligned}
$

$
\begin{aligned}
x_n \ne 0, x_{n-1} \ne 0.
\end{aligned}
$

Prove:

$
\begin{aligned}
\frac{A_{n-1}}{A_n} < \frac{B_{n-1}}{B_n} 
\end{aligned}
$ if and only if $a > b$.

\section* {1970/3.}

The real numbers $a_0, a_1, ..., a_n, ...$ satisfy the condition

$
\begin{aligned}
1 = a_0 \le a_1 \le a_2 \le \cdots \le a_n \le \cdots .
\end{aligned}
$

The numbers $b_1, b_2, ..., b_n, ...$ are defined by

$
\begin{aligned}
b_n = \sum_{k=1}^{n} \left( 1 - \frac{a_{k-1}}{a_k} \right)\frac{1}{\sqrt{a_k}}.
\end{aligned}
$
\begin{enumerate}
\item Prove that $0 < b_n < 2$ for all $n$.
\item Given $c$ with $0 < c < 2$, prove that there exist numbers
$a_0, a_1, ...$ with the above properties such that $b_n > c$
for large enough $n$.
\end{enumerate}

\section* {1970/4.}

Find the set of all   positive integers $n$ with the property that the set

$
\begin{aligned}
\{n, n+1, n+2, n+3, n+4, n+5\}
\end{aligned}
$ can be partitioned into two sets such that the product of the numbers in one set equals the product of the numbers in the other set.

\section* {1970/5.}

In the tetrahedron $ABCD$, angle $\angle BDC$ is a right angle. 
Suppose that the foot $H$ of the perpendicular from $D$ to the plane $ABC$ is the intersection of the altitudes of $\triangle ABC$. Prove that

$
\begin{aligned}
(AB + BC + CA)^2 \le 6(AD^2 + BD^2 + CD^2).
\end{aligned}
$
\\
For what tetrahedra does equality hold?

\section* {1970/6.}

In a plane there are $100$ points, no three of which are collinear.
Consider all possible triangles having these points as vertices.
Prove that no more than $70\%$ of these triangles are acute-angled.

\end{document}
